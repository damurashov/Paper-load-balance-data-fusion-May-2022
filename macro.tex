% Naming conventions
% <fn><m|l><NAME>
% fn - prefix to distinguish newly-defined commands. Stands for "function"
% m - stands for "math". Indicates that it is used in the context of a $math expression$.
% l - stands for "listing"

% Alternatives (nodes)
\newcommand{\fnmalts}{V_{N,1}, V_{N,2},\ldots}

% Aspects (nodes)
% arg 1 - indentifier of the level the nodes pertain to
\newcommand{\fnmasps}[1]{V_{#1,1}, V_{#1,2}, \ldots}

% Root node
\newcommand{\fnmroot}{V_{0,1}}

% Node
% arg 1 - level
% arg 2 - id
\newcommand{\fnmnode}[2]{V_{#1,#2}}

% Assessment in an aspect of a node, vector of weights
% arg 1 - level
% arg 2 - node (the aspect)
\newcommand{\fnmwasp}[2]{W_{#1}^{(#2)}}

% Indexed set of elements
% arg 1 - elements
% arg 2 - subscription index
\newcommand{\fnmsetind}[2]{\{ #1 \}_{#2}}

% Set of weights
% arg 1 - level for alternatives
% arg 2 - level for context
\newcommand{\fnmSetWasp}[2]{
    \fnmsetind
        % Content of  a set
        {\fnmwasp{#1}{\fnmnode{#2}{vert_{ #2 }}}}
        % Index
        {vert_{ #2 } \in L_{#2}}
}

% Short for for AHP
% arg1 - arg. #1 for AHP
% arg2 - arg. #2 for AHP
\newcommand{\fnmAhp}[2]{
    \mathbf{AHP}(#1, #2)
}

% Range
% arg 1 - from
% arg 2 - to
\newcommand{\fnmrange}[2]{\overline{#1, #2}}

% Call a procedure
% arg 1 - procedure name
% arg 2 - arguments
\newcommand{\fnlCall}[2]{$\gets\textsc{#1}(#2)$}