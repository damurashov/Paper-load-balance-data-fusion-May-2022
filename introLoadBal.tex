% TODO: the following chunk is probably should be subjected to serious amendments

It would be a far-fetched statement to say that we know the answer to that in terms of practically applicable group control. But we believe that we have to offer a mechanism for simulating group control that possesses some of the aforementioned properties such as flexibility and separation of strategic and tactical decision making while having them both within the same process. Most likely, the approach is not yet practically applicable in its current form, but we hope that further development of the idea will make practical results come to fruition.  The approach to simulating group control that we mentioned is based on using a weighted graph of preference hierarchy higher levels of which "map" to strategic objectives, while lower levels do so to tactical ones. Nodes of the graph which we call "aspects" are interpreted as parts of the general objective; edges' weights represent preference of a child node that is connected by the edge in the \textit{context} of the parent aspect.

We are by no means the first to offer using preference hierarchies to model decision-making in multi-agent systems
\cite{cartvehishvili-2018-model}, \cite{drakaki-2018-intelligent}, neither are we in regard that we offer to tweak
weights of that graph hierarchy to adjust agents' behavior \cite{zytniewski-2016-application},
\cite{brintrup-2010-behaviour}. But to our best knowledge we are the first who offered to take advantage of that graph
strcture and interpret it in the way that enables a translation of strategic objectives into tactical and influence of
high-level strategic objectives to low-level decision making done by individual agents in the context of simulating a
group coordination. And we have failed to find a similar implementation that performs weights adjustment in a
distributed manned (which we will explain further). These two aspects of our work constitute, as we hope, the novelty
that creates a modest advance in the realm of models for simulating group coordination.

The aim of this research is to prototype the idea that has been formulated in the previous work
\cite{murashov-2021-ahp}. Since it is written in russian, we re-explain some parts that describe the mechanism in
order to give you the entire picture.

%TODO: references
The structure of the paper reads as follows. First, we explain the idea and offer an algorithm implementing it. We
created a simulation based on the algorithm to have a proof of concept, so the next part is devoted to the details
regarding how the simulation is implemented. And then we give a brief overview of the simulation's results, draw some
conclusions, and offer ways of its further development.

We open-sourced the code for the simulation \cite{github}. You are encouraged to examine, fork, modify, and use
it in your projects as you see fit.
