After the generic description and the main idea of the approach have been explained, let us suggest the algorithm for
assessing the alternatives $\fnmalts$ in the context of node $\fnmroot$. And we start with laying out some definitions,
constraints, and inputs most of which are intuitive but should be stated nonetheless.

\begin{itemize}
    % TODO: context
    \item Local graph is a weighted oriented graph;
    \item It has exactly one vertex without a parent (root); it also has vertices without children (alternatives);
    \item For any 2 edges $(a,b)$, $(a,c)$, there exist no paths between $b$ and $c$ (so the local
        graph can be divided into \textit{levels});
    \item A node's \textit{level} is the max. number of edges that should be traversed from the root $\fnmroot$ to
        achieve this node, or one of its peers;
    \item There exists a path to any alternative from any node (so we can gradually build the
        resulting vector level-by-level);
\end{itemize}

Here are notations used:

\begin{itemize}
    \item $\fnmnode i j$ denotes node $j$ of level $i$;
    \item $\fnmwasp k {\fnmnode i j}$ denotes a weights vector representing assessment of all nodes of level
        $k$ in the context of node $\fnmnode i j$;
    \item $L_i$ denotes an index set for nodes of level $i$.
\end{itemize}

As an input, we have a preference graph (local graph) $\left< V, E, W \right>$ ($V$ - nodes, $E$ - edges, $W$ - weights)
having $N$ levels (it is assumed that $N>2$, otherwise the problem is trivial).
Weights for intermediate levels and the root are predefined, i.e. we have a set
$\fnmsetind
    {\fnmwasp{l}{\fnmnode{l-1}{i}}}
    {i \in L_{l-1}, l = \fnmrange{1}{N-1}}
$.

In the result, it is required to calculate $\fnmwasp N \fnmroot$ vector representing weights for actions adjusted for
strategic preferences, and assign an agent whatever action has the maximum weight. To achieve that we propose to use
\textit{AHP} algorithm which enables one to assess alternatives using preference weights of 2 intermediate levels:
%TODO: AHP, reference

\begin{equation}
    \fnmwasp a {\fnmnode {a-2} x} =
    \fnmAhp
        {\fnmSetWasp{a}{a-1}}
        {\fnmwasp {a-1} {\fnmnode {a-2} x}}.
\end{equation}

This property of the \textit{AHP} algorithm means that once lower-level weights (tactical assessment) have been
estimated, we can calculate vector $\fnmwasp N \fnmroot$ by iteratively assessing the alternatives in contexts of
vertices of levels $N-2$, $N-3$, and so on up to the root $\fnmroot$. It is possible since every alternative is
reachable from any node, so we can assess alternatives in any context. So, when strategic reasoning, iterative weights
assessment, and action selection based on this assessment are put together, the resulting algorithm for an agent
takes the following form:

\begin{algorithm}
    \caption{Agent's reasoning process}
    \label{lst:agent-reason}

    \begin{algorithmic}[1]
        \State \textbf{Input:} $\fnmsetind {\fnmwasp{l}{\fnmnode{l-1}{i}}} {i \in L_{l-1}, l = \fnmrange{1}{N-1}} $
        \State \textbf{Input:} \textbf{procedure} \fnlCall{AHP}{2}
        \\
        \Procedure{assess}{$\fnmSetWasp{N}{k}$, k}
            \If{k = 1} \Comment{The root node has been reached}
                \State {\textbf{return} \fnlCall{AHP}{$\fnmSetWasp{N}{k}$, $\fnmwasp{1}{\fnmroot}$}}
            \Else
                \State {\textbf{return} \fnlCall{assess}{$\big\{ \fnlCall{AHP}{\fnmSetWasp{N}{k}, \fnmwasp{k}{\fnmnode{k-1}{i}}} \big\}_{i \in L_{k-1} }$, k-1}}
            \EndIf
        \EndProcedure
    \end{algorithmic}
\end{algorithm}