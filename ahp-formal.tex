After the generic description and the main idea of the approach have been explained, let us suggest the algorithm for
assessing the alternatives $\fnmalts$ in the context of node $\fnmroot$. And we start with laying out some definitions,
constraints, and inputs most of which are intuitive but should be stated nonetheless.

\begin{itemize}
    % TODO: context
    \item Local graph is a weighted oriented graph;
    \item It has exactly one vertex without a parent (root); it also has vertices without children (alternatives);
    \item For any 2 edges $(a,b)$, $(a,c)$, there exist no paths between $b$ and $c$ (so the local
        graph can be divided into \textit{levels});
    \item A node's \textit{level} is the max. number of edges that should be traversed from the root $\fnmroot$ to
        achieve this node, or one of its peers;
    \item There exists a path to any alternative from any node (so we can gradually build the
        resulting vector level-by-level);
\end{itemize}

Here are notations used:

\begin{itemize}
    \item $\fnmnode i j$ denotes node $j$ of level $i$;
    \item $\fnmwasp k {\fnmnode i j}$ denotes a weights vector representing assessment of all nodes of level
        $k$ in the context of node $\fnmnode i j$;
    \item $L_i$ denotes an index set for nodes of level $i$.
\end{itemize}

As an input, we have a preference graph (local graph) $\left< V, E, W \right>$ ($V$ - nodes, $E$ - edges, $W$ - weights)
having $N$ levels. In the result, it is required to calculate $\fnmwasp N \fnmroot$ vector representing weights for
actions adjusted for strategic preferences, and assign an agent whatever action has the max. weight.
