In \cite{murashov-2022}, an agent would receive an input comprising preference hierarchy graph with pre-defined weights, the task itself, and meta-information regarding required technical capabilities a runner must possess to be able to run the task.
Then it would compare its own state with that of its neighbours, assess those in the context of criterion used by the cluster, and make a pass-or-keep decision.
The set of those criterion consists of performance metric, energy efficiency, and load distribution uniformity.

Since the goal here is to achieve separation of local and global decision making, this preference hierarchy must be extended by adding additional levels.
% TODO ref
We propose to do so by decomposing performance end efficiency.
Arguably, all other conventional load balancing quality measures like cost of maintenance or task distribution are derivatives of these two.
However, the PHG we propose (Figure \ref{fig:prefgraphLb}) is by no means comprehensive.
It only serves illustrative purposes.
