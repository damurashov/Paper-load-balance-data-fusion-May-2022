Multi agents systems and comlexity that goes along can be a source of a great excitement. It is remarkable how a system
comprising a set of agent with very limited capabilities to reason and decision making produces outstandingly good
solutions to very difficult problems. From global economy to simulated ant colonies, small agents applying their best
judgement and acting in their own best interst get things done, sometimes as gracefully as no well organized and
well-financed think tank on the entire Earth could ever do. It has been known quite a long time ago that a beautiful
whirl of a seemingly chaotic swarm of mostly independent agents can make solutions emerge.

Have we had solutions just appear by themselves, this world would not be half as spectacular. Need and absense of
low-hanging answers to our problems stimulate us to form structures and impose objectives on them, and as for multi
agent ones we have two ways to go. The first is to set agents up and let things happen. As appealing as it is and as
%TODO: "as it can be": ant colonies, ANNs
effective as it can be, this approach does not always deliver the outcomes we want. The second one is to create a rigid
system of absolute control. When creating a PID controller for a thermostat, this is exactly what we want. But for
bigger systems, this approach is prone to fail miserably, and in systems involving people, it does so with dire
consequences.
%TODO: ref. China, USSR, Venesuela, Cuba,