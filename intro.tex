Multi agent systems and comlexity that goes along can be a source of a great excitement. It is remarkable how a system
comprising a set of agent with very limited capabilities to reasoning and decision making produces outstandingly good
solutions to very difficult problems. From global economy to simulated ant colonies, small agents applying their best
judgement and acting in their own best interst get things done, sometimes as gracefully as no well organized and
well-financed think tank on the entire Earth could ever do. It has been known for quite a long time that a beautiful
whirl of a seemingly chaotic swarm of mostly independent agents are able to make solutions emerge.

But have we had solutions just appear by themselves, this world would not be half as spectacular. The necessity to solve
problems and absense of low-hanging answers to our problems stimulate us to form structures and impose objectives on
them, and as for multi agent ones we have two ways to go. The first is to set agents up and let them figure out a
strategy. As appealing as it is and as effective as it can be \cite{dorigo-2006}, we sometimes need something of a
bigger certainty meaning that we want to express the objective in a more straightworward way. The second one is to
create a rigid system of absolute control. When creating a PID controller for a thermostat, this is exactly what we
want. But for bigger and more complex systems, this approach may fall due to lack of flexibility.

The advantage of options in between those two extremes is arguably the flexibility it has. We can roughly delineate two
aspects in the process of that flexible group decision making: strategic and tactic. The first one is all about
so-called "big picture". Some entity that is entitled to establish the direction of the group's strategy, given that it
has sufficient information, can reason in bigger-scale terms and define the general course of action for the entire
group. However, it cannot (and should not) oversee all the details of the strategy's particular implementation
instances. It is better to delegate this part of decision making to smaller groups or individual agents. There is no
one-size-fits-all solutions in systems comprising multiple entities, and some decisions are better made in an ad-hoc
manner.
% TODO: "some" what?
Situational awareness \cite{endsley-1995} of individual agents may and often does possess very valuable information that
sure should be considered before a decision is made.

The question is how do we reconcile the necessity to adhere to the global strategy, and yet at the same time do not make
the system so rigid so individual agents become unable to apply the best of their situational awareness to make the best
solution possible within the confinements of the general strategy. How do we translate our general objectives into
individual agents' decision making processes?

% TODO: the following chunk is probably should be subjected to serious amendments

It would be a far-fetched statement to say that we know the answer to that in terms of practically applicable group control. But we believe that we have to offer a mechanism for simulating group control that possesses some of the aforementioned properties such as flexibility and separation of strategic and tactical decision making while having them both within the same process. Most likely, the approach is not yet practically applicable in its current form, but we hope that further development of the idea will make practical results come to fruition.  The approach to simulating group control that we mentioned is based on using a weighted graph of preference hierarchy higher levels of which "map" to strategic objectives, while lower levels do so to tactical ones. Nodes of the graph which we call "aspects" are interpreted as parts of the general objective; edges' weights represent preference of a child node that is connected by the edge in the \textit{context} of the parent aspect.

We are by no means the first to offer using preference hierarchies to model decision-making in multi-agent systems
\cite{cartvehishvili-2018-model}, \cite{drakaki-2018-intelligent}, neither are we in regard that we offer to tweak
weights of that graph hierarchy to adjust agents' behavior \cite{zytniewski-2016-application},
\cite{brintrup-2010-behaviour}. But to our best knowledge we are the first who offered to take advantage of that graph
strcture and interpret it in the way that enables a translation of strategic objectives into tactical and influence of
high-level strategic objectives to low-level decision making done by individual agents in the context of simulating a
group coordination. And we have failed to find a similar implementation that performs weights adjustment in a
distributed manned (which we will explain further). These two aspects of our work constitute, as we hope, the novelty
that creates a modest advance in the realm of models for simulating group coordination.

The aim of this research is to prototype the idea that has been formulated in the previous work
\cite{murashov-2021-ahp}. Since it is written in russian, we re-explain some parts that describe the mechanism in
order to give you the entire picture.

%TODO: references
The structure of the paper reads as follows. First, we explain the idea and offer an algorithm implementing it. We
created a simulation based on the algorithm to have a proof of concept, so the next part is devoted to the details
regarding how the simulation is implemented. And then we give a brief overview of the simulation's results, draw some
conclusions, and offer ways of its further development.

We open-sourced the code for the simulation \cite{github}. You are encouraged to examine, fork, modify, and use
it in your projects as you see fit.
