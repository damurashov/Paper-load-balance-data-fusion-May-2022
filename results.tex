We have created a simulation where we generate agents in a random manner (Table \ref{tab:sim-parameters}). The setup for
simulation has been saved and published in the repository \cite{github}, so you can re-run the simulation or generate a
new one.
%TODO: table reference

\begin{table}[h!]
    \centering
    \caption{Simulation parameters}
    \label{tab:sim-parameters}

    \begin{tabular}[hbt!]{c c}
        Parameter & Value\\
        \hline
        Number of $Hitter$s overall & 50 \\
        Number of $Resource$s overall & 20 \\
        Number of teams overall & 2 \\
        Number of $Hitter$s in "this" team & 17 \\
        Space & $([0;8] \subseteq \mathbb{R} \times [0;8] \subseteq \mathbb{R}$, manhattan distance $)$ \\
        Agents' position generator & Uniform for both dimensions \\
        $Hitter$s' energy generation & $\mathcal{N} (5,1)$ for both dimensions\\
        $Resource$s' energy generation & $\mathcal{N} (5,1)$ for both dimensions\\
        $P_{GnEnWt}$ & 0.02\\
        $P_{LsenMv}$ & 0.05\\
        $P_{Sp}$     & 0.2\\
        $P_{LsEnAg}$ & 0.05\\
        $P_{GnEnWn}$ & 0.6\\
        $P_{GnRsWn}$ & 0.3\\
        $P_{LsRsLs}$ & 2\\
        $P_{GnEn}$   & 0.5\\
        $P_{GnRs}$   & 0.5\\
        \hline
    \end{tabular}
\end{table}

For this experiments, we were iteratively increasing the ratio of "secure" strategy weight vs that of "invasive". For
each value, we performed the iteration described in Listing \ref{lst:agent-reason} omitting, of course, the execution
stage, and calculated the number of agents for each selected action (Figure \ref{fig:experiment}).

\begin{figure}[hbt!]
    \centering
    \includegraphics[width=\linewidth]{experiment.eps}

    \caption{\small The experiment's result}
    \label{fig:experiment}
\end{figure}

It is clear from the plot that as we increase the "secure-to-invasive" ratio, the amount of agents resorting to safe
behavior increases as well. Regardless of the defined strategy, the tactical assessment of the situation (i.e. weights)
remains the same.
