It is remarkable how a system comprising a set of agents with very limited capabilities to reasoning and decision making produces outstandingly good solutions to very difficult problems.
From global economy to simulated ant colonies, small agents applying their best "judgement" and acting in their own "interst" make solutions emerge from a seemingly chaotic whirl.

In creating multi agent systems, there can be delineated two approaches.
The first one is to set agents up and let them figure out a strategy through a series of interactions bearing significant information exchange while optimizing against some objective.
As appealing as it is and as effective as it can be \cite{dorigo-2006}, we sometimes need something of a bigger certainty meaning that we want to express the objective in a more straightworward way.
The second one is to create a rigid deterministic system, e.g. rule-based.
But to make such systems practical, it requires creating knowledge bases of an immense scale \cite{lenat-2022}.

There is value in a model that enables independent action while allowing to control the swarm.
A method enabling separation of global and local reasoning would make it possible to detach the implementation of an individual agent from any framework imposed by control system.

In this paper, we propose an agent-based algorithm for load balancing in heterogenous networks that has two distinctive features.
First, it enables achieving load balance in a network of an arbitrary topology automatically, without any interventions on behalf of an entity maintaining the network, and regardless of the state the network is in by the moment the process of load balancing gets started.
Second, it allows separation of load balancing objectives into two levels: strategic and tactic.
This kind of separation loosens the influence of strategic objectives on  agents' (nodes) individual decision making, thus enabling the latter to handle local objectives in a more tailored fashion, and, from engineeric point of view, loosens constraints on reasoning algorithms implemented by agents.
