Multi agent systems and comlexity that goes along can be a source of a great excitement.
It is remarkable how a system comprising a set of agents with very limited capabilities to reasoning and decision making produces outstandingly good solutions to very difficult problems.
From global economy to simulated ant colonies, small agents applying their best "judgement" and acting in their own "interst" get things done.
It has been known for quite a long time that a beautiful whirl of a seemingly chaotic swarm of (semi-) independent agents are able to make solutions emerge.

But have we had solutions just appear by themselves, this world would not be half as spectacular.
The necessity to solve problems and absense of low-hanging answers to our problems stimulate us to form structures and impose objectives on them, and as for multi agent ones we have two ways to go.
The first is to set agents up and let them figure out a strategy through a series of interactions bearing significant information exchange.
As appealing as it is and as effective as it can be \cite{dorigo-2006}, we sometimes need something of a bigger certainty meaning that we want to express the objective in a more straightworward way.
The second one is to create a rigid system of absolute control.
When creating a PID controller for a UAV's stabilization algorithm, this is exactly what we want.
But for bigger and more complex systems, this approach may fall due to lack of flexibility.

The advantage of options in between those two extremes is arguably the flexibility those have. We can roughly delineate two
aspects in the process of that flexible group decision making: strategic and tactic. The first one pertains to
so-called "big picture". Some entity that is entitled to establish the direction of the group's strategy, given that it
has sufficient information, can reason in bigger-scale terms and define the general course of actions for the entire
group. However, it cannot (and should not) oversee all the details of the strategy's particular implementation
instances. It is better to delegate this part of decision making to smaller groups or individual agents. There is no
one-size-fits-all solutions in systems comprising multiple entities, and some decisions are better made in an ad-hoc
manner.
% TODO: "some" what?
Situational awareness \cite{endsley-1995} of individual agents may and often does possess very valuable information that
sure should be included into the decision making process.

The question is how do we reconcile the necessity to adhere to the global strategy, and yet at the same time do not make
the system so rigid so individual agents become unable to apply the best of their situational awareness to make the best
solution possible within the confinements of the general strategy. How do we translate our general objectives into
individual agents' decision making processes?
